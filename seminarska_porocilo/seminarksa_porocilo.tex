\documentclass[12pt]{article}

\usepackage[T1]{fontenc}
\usepackage[utf8]{inputenc}
\usepackage[slovene]{babel}
\usepackage{lmodern}
\usepackage{amssymb}
\usepackage{amsmath}
\usepackage[tmargin=1in,bmargin=1in,lmargin=1in,rmargin=1in]{geometry}
\usepackage[backend=bibtex,sorting=none]{biblatex}
\usepackage{makecell}
\usepackage{colortbl}
\usepackage{float}
\usepackage{graphicx}
\usepackage[colorlinks]{hyperref}

\addbibresource{lit.bib}

\hypersetup{
	citecolor=blue
}

\renewcommand\arraystretch{1.5}\setlength\minrowclearance{2.4pt}

\begin{document}
	
	\begin{center}
		\large
		\thispagestyle{empty}
		UNIVERZA V MARIBORU \\
		FAKULTETA ZA ELEKTROTEHNIKO, \\
		RAČUNALNIŠTVO IN INFORMATIKO
		
		\vspace{7cm}
		
		Gregor Perčič, Matic Lukežič
		
		\vspace{1cm}
		
		\textbf{\LARGE DRUGA SEMINARSKA NALOGA}
		
		\vspace{1cm}
		
		seminarska naloga
		
		\vspace{9cm}
		
		Maribor, januar 2022
		
	\end{center}
	
	\cleardoublepage
	\pagenumbering{arabic}
	
	{\hypersetup{hidelinks}\tableofcontents}
	
	\section{Uvod}
	
	Kalkulator sva razvijala v programskem jeziku Python. Za ta jezik sva se odločila zato,
	ker je preprost in ima veliko različnih funkcij in knjižnic, ki so nama bile v pomoč pri
	izdelavi kalkulatorja.
	
	Za izgled kalkulatorja oz.~za grafični vmesnik sva uporabila knjižnico Tkinter. Vključena je
	v standardne namestitve Pythona. Z vključitvijo Tkinter knjižnice sva lahko v najino okno (okvir),
	dodajala gumbe in zaslone za izpis.
	
	\section{Aritmetika}
	
	Najprej je bilo treba napisati kodo za kalkulator, ki vsebuje osnovne računske operacije, kot so
	seštevanje, odštevanje, množenje in deljenje. Na srečo v jeziku Python obstaja funkcija \texttt{eval} (ang.~evaluate),
	ki nam izračuna podan račun. Vhod v funkcijo je niz števil, funkcija pa nato izračuna in nam vrne vrednost niza. 
	Toraj v funkcijo ne dajemo navadnih števil, ampak jih moramo pretvoriti v tip nizov (ang.~string).
	Ima tudi možnost prepoznavanja semantičnih napak, kot je na primer deljenje z nič.\cite{error} Nato sva še dodala funkcije za
	računanje korenov, potenciranje ter modul ali deljenje s celoštevilskim ostankom. Python sam po sebi ne zna
	računati korenov ali potencirati števil, zato sva morala uvoziti knjižnico \texttt{math}. Na začetku sva imela samo eno
	vrstico za vnos računa ter izpis rezultata, in da bi kalkulator bil preglednejši, sva dodala še eno vrstico.
	Torej, ko uporabnik vnaša številke, jih program vnaša v spodnjo vrstico, kadar pa želi dobiti rezultat, klikne
	na gumb \texttt{=}, s katerim program izračuna vrednost računa oz.~izraza, ga nato izpiše v zgornji vrstici, v spodnji pa
	se izpiše rezultat. Spremenila sva tudi, kako se račun prikazuje na zaslon. Primarno se vsaka posamezna črka vpisuje
	od leve proti desni, poravnava je v levo, midva pa sva naredila, da se številke vpisujejo od desne proti levi,\cite{rtol}
	poravnava pa je v desno.
	
	
	\section{Pretvarjanje med številskimi sistemi}
	
	\section{Logična vrata}
	
	\section{Branje iz datoteke}
	
	Zadnja točka je bila branje datoteke z računi in jih rešiti. Imamo neko datoteko z računi na disku,
	uporabnik pa to datoteko odpre v programu, in program mu izpiše rezultate računov. Za delo z datotekami v 
	Pythonu sva morala dodati modul \texttt{filedialog} iz knjižnice Tkinter, s katerim sva lahko odpirala, zapirala ter
	brala iz datotek. Najprej sva si morala v kalkulatorju oz.~v programu narediti novi zavihek Branje iz datotek.
	To sva naredila zato, da je program preglednejši in je posledično z njim lažje delati. Ko sva naredila svoj zavihke za
	delo z datotekami, sva v zavihku dodala zaslon, na katerega se izpiše vsebina datoteke ter rezultati, gumb s 
	katerim lahko izbrava datoteko z diska, in še gumb, ki počisti zaslon. Ko uporabnik klikne na gumb
	za odpiranje datoteke, se kliče funkcija \texttt{odpri\_txt}. Funkcija nato odpre raziskovalca (ang.~File Explorer) v
	operacijskem sistemu Windows, kjer si lahko uporabnik izbere, iz katere datoteke bi rad prebral račune. Omejila sva
	tudi datoteke, da so lahko samo tipa \texttt{.txt}, tj., da so tekstovne. Ko uporabnik izbere svojo datoteko, začne funkcija\cite{read} brati posamezne vrstice v datoteki in izpisovat rezultate. Ko pridemo do zadnje vrstice, v ozadju zapremo datoteko.

%	[2]https://stackoverflow.com/questions/20306726/right-to-left-text-in-tkinter -6.1.2022
%	[3]https://www.geeksforgeeks.org/read-a-file-line-by-line-in-python/ - 6.1.2022
%	[4]https://stackoverflow.com/questions/2140614/python-eval-error-suppression 12.1.2022
	
	\pagebreak
	
	\printbibliography
	
\end{document}